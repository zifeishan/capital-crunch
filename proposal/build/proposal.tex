\begin{abstract}

For many start-ups, lack of investment and capital has become the
bottleneck for development. This phenomenon inspires us to use machine
learning algorithms to find patterns in investment behavior from major
investors. We plan to use various domain-specific features to predict
which investors would potentially invest in a particular company .
This would not only reveal important information about investment
strategies and behaviors of investors, but also give startups ideas on
where to seek potential investment and how to adjust their strategies
so as to attract potential investors.

Our work is grounded in CrunchBase, an accessible knowledge base that
maintains full records of company and people information.

There are two primary goals of our work:

- To predict whether an investor would invest in a particular start-up based on textual, topological and domain-specific signals from both the investor and start-up.
- To analyze and reveal the factors that would prompt an investor to invest in startups so as to shed light on the adjustments the start-ups could make to attract more investments. 

<!-- { Our goal is to infer investment events in technology companies using various domain-specific features. This work can potentially cast insights to understanding factors affecting investment behaviors, making strategies for companies and investors, and mining interesting patterns happening in the market. Our work is grounded in CrunchBase, an accessible knowledge base that maintains full records of company and people information.

We ask questions such as: what factors play the most critical role in investments? Can we infer whether an investor will invest a certain company by textual, topological and domain-specific signals?}
 -->

\end{abstract}

\section{Dataset}\label{dataset}

We use data from \emph{CrunchBase.com}, one of the biggest databases
about information of companies. The current CrunchBase dataset includes
214,290 companies and 286,659 people.

\subsection{Accessing data}\label{accessing-data}

CrunchBase provides indexing data and an API for full access of their
data, yet the API has limited throughput. Due to the limitation, by now
we would like to sub-sample the dataset, and we may get the full dataset
in the future.

For data sub-sampling, a possible strategy is random sampling, where we
randomly take out a certain portion of the data. However, it would have
the potential drawback that we might not achieve consistency amongst
different parties. For example, if we sampled Facebook but not some
other companies where Mark Zuckerberg has been CEO, then we would lose
some information with regard to the network. Therefore we propose to
adopt the strategy where we start with a ``seed set'' of companies, and
get all related people and organizations in an iterative way to grow the
network.

\subsection{Data format}\label{data-format}

In terms of data format, CrunchBase provides a complete index of the
people and organizations, which includes a unique identifier that we can
use to get detailed data for people and organizations via API. The
detailed data format is demonstrated below.

The people data are like this:

\begin{verbatim}
"data": {
  "uuid": "a01b8d46d31133337c34aa3ae9c03f22",
  "properties": {
    "bio": ...
    "last_name": ...
    "first_name": ...
    ...
  }
  "relationship": {
    "degrees": {...}
    "experiences": {...}
    "news": {...}
    ...
  }
}
\end{verbatim}

The organization data for startups are like this:

\begin{verbatim}
"data": {
  "properties": {
    "description": {...}
    "founded_on": {...}
    "name": {...}
    "number_of_employees": {...}
  }
  "relationships": {...}
  "borad_members_and_advisors": {...}
  "acquisitions": {...}
  "competitors": {...}
  ...
}
\end{verbatim}

With these data as input, we construct models and run machine learning
algorithms to get predictions on investments. The section below
articulates our proposed model.

\section{Proposed Model}\label{proposed-model}

\subsection{Data Model}\label{data-model}

The crunchbase dataset has a variety of entities: \textbf{organization},
\textbf{person}, \textbf{product}, etc. There are also different
relations including \textbf{investment}, \textbf{acquisition},
\textbf{degree}, \textbf{founder}, etc.

For simplification, we categorize \textbf{organizations} into
\textbf{startups} and \textbf{investors}, and we care about predicting
\textbf{investment} relationship between them.

The data model is defined below:

\begin{itemize}
\itemsep1pt\parskip0pt\parsep0pt
\item
  \(Startup(startupId, [attributes...])\)
\item
  \(Investor(investorId, [attributes...])\)
\item
  \(Investment(investorId, startupId, isTrue)\)
\end{itemize}

Where we use features in \(Startup\) and \(Investor\) entities to
predict \(Investment\) relations.

\subsection{Problem definition}\label{problem-definition}

Given the full \(Startup\) relation and \(Investor\) relation, predict
\(isTrue\) value in \(Investment\) table, which determines if any given
investor invests a startup.

TODO

\subsection{Labeled data}\label{labeled-data}

We take ground truth investments in CrunchBase as positive training
examples

TODO

\subsection{Proposed Features}\label{proposed-features}

A rich set of features can be applied to predict investments. They may
include:

\begin{itemize}
\itemsep1pt\parskip0pt\parsep0pt
\item
  textual features: company descriptions, biography of people.
\item
  TODO
\end{itemize}

\subsection{Baseline and Oracle}\label{baseline-and-oracle}

A naive baseline model would be a random predictor that predict random

We propose to apply a factor graph model that correlates features across
this graph. Specifically, TODO

TODO baseline: logistic regression

\section{Evaluation}\label{evaluation}

We want to use the information from people and organizations, finding
the connections, common points and all the relationship we can get from
the data to learn a graph. Using this graph, we want to predict the
probability of the investors a company, and what kind of investors,
which exactly investor will invest on a certain kind of company.The data
we get from crunchbase will be separated into two parts: training and
testing, to help us evaluate the behavior of our model and help us chose
good predictors and relationship between nodes (including people and
organizations). Therefore we could evaluate if our model is trained well
and use the information we get properly.

\section{Challenges}\label{challenges}

Data sparsity. In our dataset, some investors might be totally unrelated
to some companies. A naive predictor would just predict them to be ``not
investing'', but we would like to delve deeper to see the possible
subtleties.

The need to integrate the various parties and their relationships into
one problem. For example we have start-ups, venture capital companies,
and people (e.g.~the founder of a start-up), we need to use a model so
as to accurately capture their relationship.

\subsection{Topics to Address
Challenges}\label{topics-to-address-challenges}

Natural Language Processing : We would use the Stanford NLP to process
the short descriptions so as to obtain information such as which area
the start-up is focusing on.

Probabilistic Graphical Models: PGM would be adopted to examine the
relationship between founders, companies, and investors.

TODO

Factor Networks: Factor networks would be used to model the whole
dataset. For example, degree would be an edge to connect a founder of
the start-up and a founder of a venture capital company.

\section{Related Work}\label{related-work}

In the paper Recommending Investors for Crowdfunding Projects
\cite{an2014recommending}, the author discussed a methodology to match
proposals from start-ups to the potential investors on Kickstarter with
linear regression, SVM-linear, SVM-poly and SVM-RBF, with an accuracy
rate of 82\% for static data features and 73\% for dynamic data
features. Thought their features are mostly updates made to the tweets ,
number of comments and so on, we could expand the feature set to other
features such as the education background of the investor, the area of
investment the investor usually specializes in, etc.

Another paper, Predicting new venture survival: an analysis of ``anatomy
of a start-up.'' \cite{gartner1999predicting}, Gartner, Starr and Bhat,
used a discriminant analysis to classify the potentially successful
company and unsuccessful companies. Their feature sets are worth noting,
including individual characteristics of the entrepreneurs, the efforts
by entrepreneurs (i.e.~whether they actively look for resources and
help), degree of innovation and so on. Though this paper is more on the
social science side, we would like to scrutinize the feature sets so as
to explore more meaningful and insightful features. For example, we
could extend individual characteristics to how many start-ups the CEO
has founded and their past histories. In this way, we would be able to
obtain a richer set of data.
